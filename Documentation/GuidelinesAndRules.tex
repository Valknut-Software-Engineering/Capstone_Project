\documentclass{article}
\usepackage[utf8]{inputenc}
\usepackage{geometry}
\usepackage{graphicx}
\usepackage{hyperref}
\usepackage{float}
\hypersetup{
    colorlinks,
    citecolor=black,
    filecolor=black,
    linkcolor=black,
    urlcolor=black
}

 \geometry{
 a4paper,
 left=30mm,
 right=30mm,
 top=30mm,
 }

 \graphicspath{ {Images/} }

\begin{document}

	{\begingroup
		\begin{figure}[t]
			\centering
			\includegraphics[width=350px]{vseLogo.png}
		\end{figure}
		\centering

		{\Large COS 301 Capstone Project 2017}

		\vspace*{\baselineskip}

		\rule{\textwidth}{1.6pt}\vspace*{-\baselineskip}\vspace*{2pt}
		\rule{\textwidth}{0.4pt}\\[\baselineskip]

		{\Huge Vulknut Software Engineering } \\ [0.2\baselineskip]

		\rule{\textwidth}{0.4pt}\vspace*{-\baselineskip}\vspace*{2pt}
		\rule{\textwidth}{0.4pt}\\[\baselineskip]

		{\Large Guidelines and Rules } \\ [0.2\baselineskip]

		\rule{\textwidth}{0.4pt}\vspace*{-\baselineskip}\vspace{3.2pt}
		\rule{\textwidth}{1.6pt}\\[\baselineskip] %

		% \scshape %
		% A concise specification on the functional requirements  \\
		% and use cases of NavUP \\[\baselineskip]

		% \vspace*{2\baselineskip}

		\bigskip

		Compiled By \\[\baselineskip]

		\bigskip

		{\Large Bernhard Schuld - u10297902 \\\par}

		\bigskip
		\bigskip

\endgroup}

\newpage
\tableofcontents
\newpage

	This is a first draft of the document and it is subject to change - Bernhard
	\section{General}
		\begin{itemize}
			\item ALWAYS push your code to your own branch before leaving your workstation for an extended period of time (yes, even in case of fire)
			\item Once a branch has been merged, please delete it from the repo (you will see a "merged" icon next to the branch)
			\item Remember: Real devs use a Unix system
			\item Discord has a 500ms delay, which makes it kak for gaming
			\item CS $>$ Overwatch
		\end{itemize}

	\section{Git Issues}
		\subsection{General Rules}
			\begin{itemize}
				\item You do not start coding without assigning yourself to the relevant Issue.
				\item If an Issue does not exist, you create it.
				\item You do not start coding without moving the Issue to the relevant column in Zenhub
			\end{itemize}

		\subsection{Creating an Issue}
			\begin{enumerate}
				\item Go to Issues on the Github Repo
				\item Click on "New issue"
				\item Enter Title (see syntax below)
				\item Enter Description
				\item Assign person to issue if somebody specific needs to do it, otherwise leave blank if issue is going to be done in future.
				\item Add relevant labels (If no label matches, create a new one)
				\item Add the relevant milestone
				\item Add the estimated hours for completetion
			\end{enumerate}
		\subsection{Syntax}
			\begin{itemize}
				\item Title syntax: type/very-short-description
				\item type: type of issue, usually same as label, upper- or lowercase depends on the label
				\item description: VERY short description of the issue, spaces replaced by dashes (-)
				\item label: custom labels: uppercase
				\item label: default labels: lowercase
			\end{itemize}

	\section{ZenHub Board}
		\subsection{General Rules}
			\begin{itemize}
				\item Issues that have not been started yet should remain in Idle
				\item Issues that have been finished for the current sprint but that will need revision/updating in a future sprint should also remain in Idle
				\item Not all phases are relevant to all types of issues, use own discretion to move cards/issues to relevant phases
				\item HOWEVER: all issues must start in Idle, enter Implementation, wait in Done while the pull request is awaiting review, and finally end in Closed
				\item Issues regarding code must always enter Testing
			\end{itemize}

	\section{Using Git}
		\subsection{Branching}
			\begin{enumerate}
				\item git checkout dev
				\item git fetch \&\& git pull
				\item git checkout dev -b type/issue-you-are-working-on
				\item git push -u origin type/issue-you-are-working-on
			\end{enumerate}
		\subsection{Merging}
			\begin{enumerate}
				\item git add -A
				\item git commit -as
				\item Enter git message (syntax below)
				\item crl + x (terminal will ask for confirmation)
				\item Y or N
				\item git checkout dev
				\item git fetch \&\& git pull
				\item git checkout type/issue-you-are-working-on
				\item git pull origin dev
				\item At this point, the terminal will indicate if there are merge conflicts. If there are, skip to Merge Conflicts subsection
				\item git push
			\end{enumerate}
		\subsection{Pull Requests}
			\begin{enumerate}
				\item Navigate to the github repo using your browser
				\item Since you recently pushed to your own branch, there should be a "Compare \& Pull request" button. Click on it
				\item Make sure you select the correct base, should be "dev" in most instances.
				\item Git should say that the branches are "able to merge". If it doesn't then you fucked up with an earlier step as there are merge conflicts.
				\item Assign one or more reviewers to your pull request
				\item Add the "Review" label to your pull request.
				\item Add a comment if you wish, and click on "Create pull request"
			\end{enumerate}
		\subsection{Reviewing Pull Requests}
			\begin{enumerate}
				\item Double check that the branches are able to automatically merge
				\item Wait for all TravisCI tests to pass (if they have not completed already)
				\item If you want to be completely sure that the code works (because you will sign off on it, you will share responsiblity) checkout the branch that was worked on and manually test the work.
				\item Leave a comment that explains what needs to change if needed or approve the pull request
				\item You can either merge the branches yourself then, or the original pull requestee can merge the branches.
			\end{enumerate}
		\subsection{Merge Conflicts}
			\begin{enumerate}
				\item If there are any merge conflicts, your chosen editor should indicate where the coflicts are. If your editor doesnt show it, get a better editor.
				\item Resolve said merge conflicts
				\item Return to Merging subsection and repeat steps.
			\end{enumerate}
		\subsection{Syntax}
			\begin{itemize}
				\item git message sytnax:

				type(issue-you-are-working-on): Short description of changes

				Detailed description of changes
				\item type(issue-you-are-working-on) all lowercase
			\end{itemize}
\end{document}

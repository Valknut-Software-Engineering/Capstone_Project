
\section{System Requirements and Design}

\subsection{Introduction}

	\subsubsection{Purpose}
	This document serves to outline the overall description and requirements of 	the system. This document also serves as a guideline to the developers in 		order to ensure the final product meets these requirements, and indicates 		to the client what the required technologies are in order to be able to use 	this system.

	\subsubsection{Scope}

	The overall objective of this project is to provide any given user with a toolkit, with which the individual can create a 3D virtual reality presentation with ease. Our goal is to make it simple to use, enabling virtually any user to utilize the power of 3D, without having to build 3D objects completely from scratch. The user would custom build a 3D environment built upon a variety of available templates offered, or by selecting a set of 3D models and skyboxes when choosing to create a project from the ground up, taking user experience to a whole nother level.

	\subsubsection{Definitions, Acronyms and Abbreviations}
			\paragraph{MEAN}	MongoDB, Express.js, AngularJS (or Angular), and Node.js
			\paragraph{VR}	Virtual Reality
			\paragraph{MVP} Minimum Viable Product
			\paragraph{MTBF} Mean Time Between Failures
		% \section{References}

\subsection{Design}

	\subsubsection{Software Methodology}
	We will follow the Agile development methodology. The principles this methodology is based on advocates planning, constantly evolving development, early delivery and continues improvements, and it encourages flexibility as well as maintainability.

	\begin{flushleft}
	The agile development process is built on four main principles:
		\begin{enumerate}
			\item Individual and team interactions over processes and tools.
			\item Working software over comprehensive documentation.
			\item Customer collaboration over contract negotiation.
			\item Responding to change over following a plan.
		\end{enumerate}
	\end{flushleft}

\begin{flushleft}

The Agile development approach allows for frequent opportunities for clients to be involved in. Requirements are then reprioritized according to client specifications and they are elaborated on. The process of Agile development is based on the following actions:

        \begin{itemize}       
			\item Short timeboxes of iterative development.
			\item Early and repeated client/user feedback.
			\item Reprioritization of work based on the client/user so that emergent requirements can be handled.
			\item Selecting a specific approach of which there are a variety of options including, Extreme Programming, Scrum, Lean Development, and Feature-Driven Development.
		\end{itemize}

Some of the benefits of using the Agile development include stakeholder engagement, transparency, early and predictable delivery, predictable costs and schedule, allows for change, focus on the client, and ultimately improving the quality of the software.

For the above-mentioned reasons, we had chosen to utilize the Agile Software Methodology as it was the most applicable satisfying our needs as well as our client's.

	\subsubsection{Development Technique}
	During our first meeting with EPI-USE they had mentioned that we should make use of a development technique called MVP. A MVP is the most basic version of a product that can still be released. The point of this technique would be that early adopters would see the potential that the final product could offer, and give developers valuable feedback needed to guide them forward.

\subsection{System Requirements}

	\subsubsection{Functional Requirements}
	
	The following functional requirements will be met:
  	
  		\begin{enumerate}
			\item The toolkit will allow users to create 3D environments.
			\item Users will be able to add objects to the environment.
			\item The toolkit will allow users to select pre-built environments.
			\item Users will be able to share content they have created, or integrate content that is publically available, making the project evolve even further through the comunity.
		\end{enumerate}

	\subsubsection{Non-Functional Requirements}

	The following non-functional requirements will be met:

		\begin{enumerate}
  			\item Usability - key concern is to make this system easy to use.
  			\item Reliability - the system should not fail, aiming for a high MTBF. A strategy will be in place for error detection.
  			\item Portability - making use of Unity3D allows our software to be compatible with a large variety of VR devices. A simple installation is all that is required.
  			\item Modifiability - aiming for a community driven approach we will ensure that software is easily upgraded.
  			\item Platform constraints - developing in Unity3D caters for the widest VR devices.  			
		\end{enumerate}

\subsection{Target Audience Characteristics}

	Our first focused audience would be targeted at the educational sector. Our initial goal would be to provide a toolkit for an individual to build a basic educational scene.
	
\subsection{Constraints}
There are several constraints needed to be taken into consideration.
\bigskip
\begin{flushleft}
\textsl{Platform constraints:}
\end{flushleft}
	\begin{itemize}		
		\item Mono, an open source development platform based on the .NET Framework. Mono’s implementation is based on the ECMA standards for C\# and the Common Language Infrastructure.
		\item For development:
			\begin{itemize}
				\item Windows 7 SP+1, 8, 10; Mac OS X 10.8+.
			\end{itemize}
		\item For running Unity applications/games (depending on the complexity of the project):
			\begin{itemize}
				\item Windows XP SP2+, Mac OS X 10.8+, Ubuntu 12.04+, SteamOS+.
			\end{itemize}
	\end{itemize}
\medskip		
\textsl{Device hardware constraints:}
	\begin{itemize}
		\item Graphics card: DX9 (shader model 3.0) or DX11 with feature level 9.3 capabilities.
		\item CPU: SSE2 instruction set support.
	\end{itemize}
\medskip		
\textsl{Video size:}
	\begin{itemize}
		\item The exported video should be a realistic size, taking bandwidth and cap into consideration.
	\end{itemize}
\medskip	
\textsl{Community content needs to be a reasonable size (in community guidelines):}
	\begin{itemize}
		\item Contributing to the complexity of a project will increase exported video size.
	\end{itemize}
\medskip	
\textsl{Community content needs to be relatively optimized (in community guidelines):}
	\begin{itemize}
		\item Again, contributing to the complexity of a project will increase exported video size.
	\end{itemize}
\medskip
\textsl{Other constraints that will be considered and in which further research will be conducted as implementation progresses include:}
	\begin{itemize}
		\item Possible VR device constraints with regards to environment editing.
		\item Fixed set of templates.
		\item Unity assets only for community driven content.
	\end{itemize}

\subsection{Technologies}
	In our first meeting with EPI-USE we had discussed the use of various technologies. They had given us "free will" with regards to what technologies to use. After we had committed ourselves to extensive research we had selected the following, but did not limit ourselves to:
	\begin{itemize}		
		\item Creating a 3D environment to design and bring to life a 3D presentation.		
		\item Unity 3D virtual reality system tool kit library.
		\item HTC Vive virtual reality gear (already available).
		\item Import external models.
		\item Using plug and play libraries.
		\item Possibly include library templates for uses to build on.
		\item Community driven approach.
		\item Windows 10 environment.
		\item Docker.
		\item TravisCI.
	\end{itemize}

